\documentclass[11pt]{article}
\usepackage{xepersian}
\settextfont{Nazli Bold}
\setlatintextfont{Ubuntu}
\setdigitfont{Ubuntu}
\title{آموزش زبان برنامه نویسی Ruby}
\author{محمدرضا حقیری}
\begin{document}
\maketitle
\newpage{}
\tableofcontents
\newpage{}
\section{مقدمه}
زبان روبی ، یک زبان اسکریپتی، شی گرا، تابعی و مدرن است. این زبان توسط یوکیهیرو ماتسوموتو
\LTRfootnote{Yukihiro Matsumoto}
ایجاد شده و هم اکنون توسعه دهندگانی از سراسر جهان، آن را توسعه میدهند. این زبان، نحو
\LTRfootnote{Syntax}
ساده ای داشته، و می توان در عرض چند روز آن را به خوبی فرا گرفت. این زبان، برای توسعه برنامه های کاربردی دسکتاپ، وب و حتی نوشتن کتابخانه و سرویس های مختلف، کاربرد دارد. شما میتوانید به سادگی و با استفاده از این کتاب، این زبان را بیاموزید. 
\subsection{این کتاب برای چه کسانی است؟}
به اختصار، باید گفت همه علاقمندان به یادگیری این زبان ساده و جذاب. جواب طولانی تر، اینست که این کتاب، برای کسانی است که واقعا میخواهند به سراغ روبی بروند، و چون روبی مرجع فارسی مطمئنی ندارد، پشیمان می شوند. پس اگر از هر دو دسته هستید، توصیه میکنم این کتاب را از دست ندهید. 

\subsection{آیا میتوانیم این کتاب را به اشتراک بگذاریم؟}
این کتاب، رایگان عرضه شده است، همچنین شما نیز میتوانید «بدون قصد تجاری»، آن را با دوستان و خانواده خود به اشتراک بگذارید، یا برای دانلود در وبلاگ و وبسایت شخصی خود قرار دهید. 
\subsection{چگونه این کتاب را بخوانیم؟}
برای خواندن این کتاب، دو راه دارید، یا صرفا مطالعه روزنامه وار انجام دهید تا به آشنایی اجمالی با نحو زبان روبی برسید، یا اینکه تک تک نرم افزارها و مراحل گفته شده در کتاب را مرحله به مرحله انجام دهید، تا کاملا به روبی مسلط شوید. 
\newpage{}


\end{document}
